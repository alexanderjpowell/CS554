\documentclass[11pt]{article}

\usepackage{times,mathptm}
\usepackage{pifont}
\usepackage{exscale}
\usepackage{latexsym}
\usepackage{amsmath}
\usepackage{amssymb}
\usepackage{amsthm}
\usepackage{epsfig}
\usepackage{tikz}
\usepackage{textcomp}
\usepackage{enumerate}

\textwidth 6.5in
\textheight 9in
\oddsidemargin -0.0in
\topmargin -0.0in

\parindent 15pt     % How much the first word of a paragraph is indented. 
\parskip 1pt	   % How much extra space to leave between paragraphs.

\begin{document}

\begin{center}             % If you're only centering 1 line use \centerline{}
\begin{LARGE}
{\bf CS 554 Homework \#2}
\end{LARGE}
\vskip 0.25cm      % vertical skip (0.25 cm)

Due: Thursday, March 3\\  % force new line
Alexander Powell
\end{center}

\begin{enumerate}
\item Message digests are reasonably fast, but here's a much faster function to compute. Take your message, divide it into 128-bit chunks, and image all the chunks together to get a 128-bit result. Do the standard message digest on the result. Is this a good message digest function?

\textbf{Solution: }

No, this is not a good message digest function because it's not hard to generate another message with the same 128-bit $\oplus$.  This is because it will have a lot of collisions.  

\item Why do MD4, MD5, and SHA-1 require padding of messages that are already a multiple of 512 bits?

\textbf{Solution: }

MD4, MD5, and SHA-1 require padding of messages that are already a multiple of 512 bits because if they didn't it would be very easy for someone to find two messages with the same hash.  For example, let's take 2 messages, $A$ and $B$.  Let's say that $A$ is the same as $B$ but padded following the MD4 standard, so that $A$ is a multiple of 512 bits.  If no padding is used for $A$, then MD4($A$) $=$ MD4($B$).   

\item What are the minimal and maximal amounts of padding that would be required in each of the message digest functions?

\textbf{Solution: }
\begin{itemize}
\item \textbf{MD2}

With MD2, the length has to be a multiple of 16 bytes.  However, padding is always done even if the message is already a multiple of 16, and in this case another 16 bytes are added.  Otherwise, the number of necessary bytes from 1 through 16 are added.  So, the minimum amount of padding required could be 1 bytes, and the max could be 16.  
\item \textbf{MD4}

In MD4, the message is padded so that it's length is a multiple of 448 (512-64), so that there are 64 bits remaining before the message length is a multiple of 512 bits.  Like MD2, padding is performed even if the message length is already a multiple of 448 bits.  So, the minimal amount of padding needed is 1 and the maximum is 512 bits.  
\item \textbf{MD5}

The padding for the MD5 message digest is identical to that of MD4.  Therefore, the minimal amount of padding needed is 1 and the maximum is 512 bits.  
\item \textbf{SHA-1}

The padding protocol for SHA-1 is the same as that of MD4 and MD5, with a small exception.  SHA-1 is not defined for a message that is longer than $2^{64}$ bits.  However, this isn't really a problem because a message of that size would take several hundred years to transmit anyway, so we assume all practical messages will be shorter in length.  
\end{itemize}

\item Assume a good 128-bit message digest function. Assume there is a particular value, $d$, for the message digest and you'd like to find a message that has a message digest of $d$. Given that there are many more 2000-bit messages that map to a particular 128-bit message digest than 1000-bit messages, would you theoretically have to test fewer 2000-bit messages to find one that has a message digest of $d$ than if you were to test 1000-bit messages?

\textbf{Solution: }

No.  In both scenarios you would still need to try all $2^{128}$ messages.  

\item For purposes of this exercise, we will define random as having all elements equally likely to be chosen. So a function that selects a 100-bit number will be random if every 100-bit number is equally likely to be chosen. Using this definition, if we look at the function ``$+$" and we have two inputs, $x$ and $y$, then the output will be random if at least one of $x$ and $y$ are random. For instance, $y$ can always be $51$, and yet the output will be random if $x$ is random. For the following functions, find sufficient conditions for $x$, $y$, and $z$ under which the output will be random:

\textbf{Solution: }

\begin{itemize}
\item $ \sim x$

If $x$ is random, then it's sufficient to say that $ \sim x$ is random as well.  

\item $x \oplus y$

If $x$ is random, then it's sufficient to say that $x \oplus y$ is also random.  Equivalently, we could say that if $y$ is random then $x \oplus y$ is also random.  

\item $x \lor y$

In this case, both $x$ and $y$ have to be random for $x \lor y$ to be sufficiently random.  This is because even if $x$ was random, $y$ could always be true, and $x$ would have no affect on the output of the function.  

\item $x \land y$

If $x$ is random, then it's sufficient to say that $x \land y$ is also random.  Equivalently, we could say that if $y$ is random then $x \land y$ is also random.  

\item $(x \land y) \lor (\sim x \land z)$  [the selection function]

In this case, $x$ needs to be random for the selection function, $(x \land y) \lor (\sim x \land z)$, to be random.  

\item $(x \land y) \lor (x \land z) \lor (y \land z)$  [the majority function]

In this case, it's sufficient for any of $x$, $y$, or $z$ to be random for $(x \land y) \lor (x \land z) \lor (y \land z)$ to be random.  

\item $x \oplus y \oplus z$

In this case, it's sufficient for just any one of $x$, $y$, or $z$ to be random so that $x \oplus y \oplus z$ is random.  

\item $y \oplus (x \lor \sim z)$

In this case, it's sufficient for $y$ to be random so that $y \oplus (x \lor \sim z)$ is random.  

\newpage

\end{itemize}
\item How do you decrypt the encryption specified in §5.2.3.2 \textit{Mixing In the Plaintext}?

\textbf{Solution: }
The decryption of $b_1, b_2, \hdots$ is shown below.  
$$ b_n = MD(K_{AB} | c_{n-1}) $$
$$ b_{n-1} = MD(K_{AB} | c_{n-2}) $$
$$ \vdots $$
$$ b_{n-i} = MD(K_{AB} | c_{n-i-1}) $$
$$ \vdots $$
$$ b_2 = MD(K_{AB} | c_1) $$
$$ b_1 = MD(K_{AB} | IV) $$

and $p_1, p_2, \hdots$ is calculated as follows:
$$ p_n = c_n \oplus b_n $$
$$ p_{n-1} = c_{n-1} \oplus b_{n-1} $$
$$ \vdots $$
$$ p_1 = c_1 \oplus b_1 $$

\item Can you modify the encryption specified in §5.2.3.2 \textit{Mixing In the Plaintext} so that instead of $b_i = MD(K_{AB}|c_{i-1})$ we use $b_i = MD(K_{AB}|p_{i-1})$? How do you decrypt it? Why wouldn't the modified scheme be as secure? (Hint: what would happen if the plaintext consisted of all zeroes?)

\textbf{Solution: }
The encryption could be modified so that $b_i = MD(K_{AB}|p_{i-1})$.  Decryption would work as follows:
$$ b_n = MD(K_{AB} | p_{n-1}) $$
$$ b_{n-1} = MD(K_{AB} | p_{n-2}) $$
$$ \vdots $$
$$ b_{n-i} = MD(K_{AB} | p_{n-i-1}) $$
$$ \vdots $$
$$ b_2 = MD(K_{AB} | p_1) $$
$$ b_1 = MD(K_{AB} | IV) $$
However, this modified scheme wouldn't be as secure because you would lose the complexity from the XOR between each $c_i$ and $b_i$.  Also, if the plaintest consisted of all zeros, it would be very easy for someone with malicious intent to figure out the message.  

\item See attached java code.  
\end{enumerate}
\end{document}



























