\documentclass[11pt]{article}

\usepackage{times,mathptm}
\usepackage{pifont}
\usepackage{exscale}
\usepackage{latexsym}
\usepackage{amsmath}
\usepackage{amssymb}
\usepackage{amsthm}
\usepackage{epsfig}
\usepackage{tikz}
\usepackage{textcomp}
\usepackage{enumerate}

\textwidth 6.5in
\textheight 9in
\oddsidemargin -0.0in
\topmargin -0.0in

\parindent 15pt     % How much the first word of a paragraph is indented. 
\parskip 1pt	   % How much extra space to leave between paragraphs.

\begin{document}

\begin{center}             % If you're only centering 1 line use \centerline{}
\begin{LARGE}
{\bf CS 554 Homework \#3}
\end{LARGE}
\vskip 0.25cm      % vertical skip (0.25 cm)

Due: Tuesday, March 15\\  % force new line
Alexander Powell
\end{center}
\begin{enumerate}

\item In mod $n$ arithmetic, the quotient of two numbers $r$ and $m$ is a number $q$ such that $mq = r \text{ mod } n$. Given $r$, $m$, and $n$, how can you find $q$? How many $q$s are there? Under what conditions is $q$ unique? [Hint: $mq = r \text{ mod } n$ iff there is an integer $k$ such that $qm + kn = r$. Divide by $\gcd(m,n)$.]

\textbf{Solution: }
To find $q$, start by dividing $r$ by $m$ mod $n$.  To do so, divide $r$ by $m$, and let's call the quotient $x$ and the remainder $r_1$.  Next, divide $n$ by $m$ and let's call the quotient $y$.  Also, lets call the remainder of this division $r_2$.  From here, solve for $a$ where $r_1 + a \times r_2$ is a multiple of $n$.  Finally, $q$ is calculated using the following equation.  
$$ q = x + a \times y $$

Also, $q$ will be unique if $n$ is a prime number, excepts for when $m = 0$.  

\item For what type of number n is $\phi(n)$ largest (relative to $n$)?

\textbf{Solution: }
$\phi(n)$ is largest for prime numbers since all numbers from $1$ to $n-1$ are relatively prime to $n$ is $n$ is prime.  

\item For what type of number n is $\phi(n)$ smallest (relative to $n$)?

\textbf{Solution: }
$\phi(n)$ is smallest for a number containing the max possible number of prime factors.  

\item Is it possible for $\phi(n)$ to be bigger than $n$?

\textbf{Solution: }
No, it is not possible for $\phi(n)$ to be bigger than $n$ because the max value of a totient function occurs when $n$ is prime and the value of the function is $n - 1$.  

\item In section §6.4.2 Defenses Against Man-in-the-Middle Attack, it states that encrypting the Diffie-Hellman value with the other side's public key prevents the attack. Why is this the case, given that an attacker can encrypt whatever it wants with the other side's public key?

\textbf{Solution: }
This is the case because the person conducting the attack isn't able to decrypt the Diffie-Hellman values that are sent to them.  Therefore, they won't be able to compute the shared message.  

\item In RSA, is it possible for more than one $d$ to work with a given $e$, $p$, and $q$?

\textbf{Solution: }
It's important to note that $d$ is the multiplicative inverse of $e \text{ mod } (p-1)\times(q-1)$.  This means that $d$ is unique modulo $(p-1)\times(q-1)$.  Also, since $e$ has a multiplicative inverse, then we know $gcd(e, (p-1)\times(q-1)) = 1$.  Therefore, there's only one element in $Z_{(p-1)\times(q-1)}$ that yields $1$ when multiplied by $e$.  So, it is not possible for more than one $d$ to work with a given $e$, $p$, and $q$?

\item In RSA, given that the primes $p$ and $q$ are approximately the same size, approximately how big is $\phi(n)$ compared to $n$?

\textbf{Solution: }
We know that $\phi(n) = (p-1)\times(q-1)$.  By expanding out this binomial, we get $pq - p - q + 1$, which is the same as $n - 2 \times \sqrt{n}$.  This can be written as the product $n\bigg(1 - \dfrac{2}{\sqrt{n}}\bigg)$.  

\item What is the probability that a randomly chosen number would not be relatively prime to some particular RSA modulus $n$?  What threat would finding such a number pose?

\textbf{Solution: }
The probability that a number is not relatively prime to $n$ can be written as
$$ \frac{(pq - (p - 1)(q - 1))}{pq} = \frac{(p + q - 1)}{pq} = \frac{1}{p} + \frac{1}{q} - \frac{1}{pq} $$
This is a threat because if such a number was found it could lead to a factorization of $n$ from which it would be possible to figure out $\phi(n)$ as well as the multiplicative inverse of any public key mod $\phi(n)$.  It would also give the attacker the ability to find the private key of a key pair.  

\item Suppose Fred sees your RSA signature on $m_1$ and on $m_2$ (i.e. he sees $m_1^d$ mod n and $m_2^d$ mod n). How does he compute the signature on each of $m_1^j$ mod $n$ (for positive integer $j$), $m_1^{-1} \mod n$, $m_1·m_2 \mod n$, and in general $m_1^j \times m_2^k mod n$ (for arbitrary integers $j$ and $k$)?

\textbf{Solution: }
The signature of $m_1^j$ mod $n$ for any $j$ positive integer can be computed by $(m_1^d)^j$ mod $n = (m_1^j)^d$ mod $n$.  

The signature of $m_1^{-1}$ is $(m_1^{-1})^{d}$ mod $n = (m_1^{-d})$ mod $n = (m_1^d)^{-1}$ mod $n$.  This can be calculated by finding the multiplicative invserse of $m_1^d$ mod $n$ using Euclid's algorithm.  

The signature of $m_1 \times m_2$ is $(m1*m2)^d$ mod $n = ((m1)^d$ mod $n) *(m2^d$ mod $n)$ mod $n$.  

Finally, to compute the signature for $m_1^j \times m_2^k mod n$ (for arbitrary integers $j$ and $k$), we know that if $j < 0$ then $j = -1 * -j$, where $j < 0$.  Therefore the signature of $m_1^j$ can be computed using the signature of $m_1^{-j}$ and the signature of $m_1^j$ can be computed using the signature of $m_1^{-j}$.  Continuing from this, we can compute the signature of $m_2^k$ because we already know how to compute the signature of $m_1^j$ and $m_2^k$.  

\item This problem illustrates the point that the Diffie-Hellman protocol is not secure without the step where you take the modulus; i.e. the ``Indiscrete Log Problem" is not a hard problem! You are Eve, and have captured Alice and Bob and imprisoned them.  You overhear the following dialog.  Bob: Oh, let's not bother with the prime in the Diffie-Hellman protocol, it will make things easier.  Alice: Okay, but we still need a base $g$ to raise things to. How about $g=3$?  Bob: All right, then my result is $27$.  Alice: And mine is $243$.  What is Bob's secret $x_B$ and Alice's secret $x_A$? What is their secret combined key? (Don't forget to show your steps.)

\textbf{Solution: }
First of all, we know $g=3$, Bob's result is $27$, and Alice's result is $243$.  So, we can write $B = 27 = 3^3$ and $A = 243 = 3^5$.  So, we now know that $x_B = 3$ and $x_A = 5$.  Also, their secret combined key is $$ 243^3 = 27^5 = 14348907. $$

\item See attached java code.  

\end{enumerate}
\end{document}



























